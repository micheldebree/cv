% On OSX, with Homebrew:
% brew install basictex
% sudo tlmgr update --self
% sudo tlmgr install a4wide
% sudo tlmgr install titlesec
% 
% pdflatex cv.tex
\documentclass[a4paper]{article}
\usepackage[dutch, english]{babel}
\usepackage{a4wide}

% set as an option: nl or en
\usepackage[nl]{optional}
%\usepackage[dvipdfm]{hyperref} % for hyperlinks in PDF
\usepackage{url}
%\usepackage{sectsty} % for setting section header styles
%\usepackage[normalem]{ulem}

\usepackage[small]{titlesec}

% turn off paragraph indenting
\setlength{\parindent}{0pt}

\newcommand{\ms}{MS}
\newcommand{\postgresql}{Pos\-greSQL}
\newcommand{\mysql}{My\-SQL}
\newcommand{\omschrijving}{Omschrijving:}
\newcommand{\activiteiten}{Activiteiten:}
\newcommand{\tools}{Tools, methoden en technieken:}

\newcommand{\tabelh}[1]{\textbf{#1}}

%\renewcommand{\rmdefault}{bch} % bitstream charter for rm
%\renewcommand{\ttdefault}{pcr} % courier for tt 
%\renewcommand{\sfdefault}{phv} % avantgarde for sf

\opt{en}{\title{\Huge R\'esum\'e}}
\opt{nl}{\title{\Huge\sffamily Curriculum Vit\ae}}
\author{\sffamily Michel de Bree}
\date{}

% available fonts:
% ppl (rm) palatino
% pag (sf) avantgarde
% phv (sf) helvetica
% ptm (rm) times
% pcr (tt) courier
% pbk (rm) bookman
% pnc (rm) new century schoolbook
% pzc (rm)
% bch (rm) bitstream charter
% universal (sf) raster

%\renewcommand{\rmdefault}{ppl} % palatino for rm
%\renewcommand{\rmdefault}{ppl} % palatino for rm

%\allsectionsfont{\sf}
%\sectionfont{\sf}
%\allsectionsfont{\usefont{T1}{pzc}{m}{n}}

%\titleformat*{\section}{\sffamily\Large\titlerule}

\titleformat{\section}{\sffamily\large}{\thesection}{.5em}{}[\titlerule]

\begin{document}
    \opt{en}{\selectlanguage{english}}
    \opt{nl}{\selectlanguage{dutch}}
{\maketitle}
%{\sf \maketitle}


% persoonsgegevens {{{
\section*{Persoonsgegevens}	
		
    \begin{tabular}{l l}
        Adres: & Thorbeckelaan 173 \\
        & 2564 BJ  Den Haag \\
        Telefoon: & 070~388~2552 \\
        Mobiel: & 0644~252~350\\
        E-mail: & \texttt{m.debree@gmail.com} \\
        Geboren : & 18 september 1972 \\
        Geslacht: & Mannelijk \\
        Nationaliteit: & Nederlandse \\
        Burgelijke staat: & Samenwonend \\
        Rijbewijs: & B \\
        Talenkennis: & Nederlands (moedertaal) \\
        & Engels (vloeiend) \\
        & Frans (redelijk) \\
    \end{tabular}
% }}}
            
% profiel {{{
\section*{Profiel}
Ervaren software ontwikkelaar met academische opleiding en denkniveau,
gespecialiseerd in web\-applicaties en --services, gebruik makend van Java/J2EE.
Ik heb een autodidactische instelling en houd mezelf voortdurend op de hoogte
van de nieuwe ontwikkelingen op het gebied van software development en
architectuur. Op basis van mijn ervaring weet ik deze snel op waarde te
schatten. Mijn interesse gaat o.a. uit naar het voortdurend verbeteren van
softwareontwikkeling als proces, zonder daarbij de pragmatische aspecten uit
het oog te verliezen.
% }}}

% algemene kennis en ervaring {{{
		\section*{\opt{en}{Acquired Skills \&\ Knowledge}\opt{nl}{Algemene
		kennis \&\ ervaring}}

		\begin{tabular}{l p{10cm}}
			\opt{en}{Methodologies \&\ Techniques}\opt{nl}{Methoden \&\
			Technieken:} & Prince 2, UML, object oriented design \&
            development, design
			patterns, refactoring, unit testing, configuration management,
            code review.\\
			\opt{en}{Programming Languages}\opt{nl}{Programmeertalen:} & Java,
			C, Visual Basic, Pascal, Delphi, Assembler \\
			Web development & Java, J2EE, JSP, XML, HTML, XHTML, CSS, PHP. \\
			Libraries \&\ Frameworks & Tapestry, Struts, MMBase, JUnit, Velocity,
			Hibernate \\
			Ontwikkelomgeving: & Eclipse, Ant, CVS, Subversion \\	
			Databases: & \mysql, \postgresql, SQLServer, HSQLDB \\
			\opt{en}{Personal Interests}\opt{nl}{Persoonlijke interesses:} &
			Java/J2EE/J2ME, open source \&\ open standards, 
			web development, component-based development,
			configuration management, softwareontwikkeling in het algemeen. \\
		\end{tabular}
% }}}

% education {{{
	\section*{\opt{en}{Education}\opt{nl}{Opleiding}}

		\begin{tabular}{l p{10cm}}
			1991 -- 2001 & 
			\opt{en}{\textbf{Information Technology, Delft University of
			Technology, Delft, NL}}
			\opt{nl}{\textbf{Technische Informatica, Technische Universiteit
            Delft}}:
			\opt{en}{Specialized in Software
			Engineering.}\opt{nl}{Afstudeerrichting Software Engineering.} \medskip \\
			
			1984 -- 1991 & \textbf{VWO, Vlietland College, Leiden}:
			\opt{en}{Final exams in maths, physics, chemistry, biology, dutch, english
			and business economics.}
			\opt{nl}{Eindexamen in wiskunde, natuurkunde, scheikunde,
			biologie, nederlands, engels en bedrijfseconomie.} \medskip \\
		\end{tabular}
% }}}		

% courses {{{
	\section*{\opt{en}{Courses}\opt{nl}{Cursussen}}
	
		\begin{tabular}{l l}
			2003 & \textbf{Prince 2 Foundation}, Imtech ICT \\
			2003 & \textbf{MMBase in a Nutshell}, MMatch \\
			2002 & \textbf{Developing J2EE Compliant Applications},
			Sun Educational Services \\
			2001 & \textbf{Object Oriented Analysis \&\ Design}, 
			Sun Educational Services \\
		\end{tabular}
% }}}		

% working experience {{{

	\section*{\opt{en}{Working Experience}\opt{nl}{Werkervaring}}
	
		\begin{tabular}{l p{10cm}}

			\opt{en}{2001 -- today}\opt{nl}{2001 -- heden} & \textbf{Imtech
			ICT Industry Solutions, Den Haag, NL}:
			Software Engineer. 
            \opt{en}{Design and implementation of software projects.}
            \opt{nl}{Ontwerpen en implementeren van software projecten.}
			
			\medskip \\
		
			1996 -- 2001 & 
			\opt{en}{\textbf{Laboratory of Physiology, Leiden University Medical Centre,
			Leiden, NL}:}
			\opt{nl}{\textbf{Laboratorium voor Fysiologie, Leids Universitair
			Medisch Centrum, Leiden}:}
			\opt{en}{I.T. related support of physiological research.  Includes development of
			online measuring/data collection software, data processing and analysis
			software, technical support, website development and maintenance.}
			\opt{nl}{I.T. gerelateerde support van fysiologisch onderzoek.
			Ontwikkeling van online meetsoftware, software voor het verwerken
			en analyseren van meetgegevens, technisch support, website
            ontwikkeling en onderhoud.}
			
			\medskip \\
	
			1994 -- 1996 &
			\opt{en}{\textbf{``Computer Ondersteund Onderwijs'', Medical
			IT, Leiden University Medical Centre, Leiden, NL}:			
			Responsible for the implementation of Egel 3.0, together with one collegue.
			Involved upgrading a large database of questions to be used in exams at the
			medical faculty and developing the software used to maintain the database,
			automatically generate exams and administer them online.  For this purpose
			\ms\ Access was used as the database, and \ms\ Visual Basic for
			the software. Egel has since developed into a commercial
			product (see \url{http://www.tokobv.nl}).}
			\opt{nl}{\textbf{Computer Ondersteund Onderwijs (COO), Medische
			Informatica, Leids Universitair Medisch Centrum, Leiden}:
            Software Engineer. Produktontwikkeling.}
			
			\medskip \\ 

			1994 & 
			\opt{en}{\textbf{Norwegian University of Science and Technology,
			Trondheim, NO}: Lived in Trondheim for a few months to do
			practical assignment for my studies, together with two fellow
			students. We built a demo application for a contraint-solving
			environment called CHIP.}
			\opt{nl}{\textbf{Norwegian University of Science and Technology,
			Trondheim, Noorwegen}:
			Samen met twee medestudenten een ``derdejaars praktikum''
			uitgevoerd in het kader van mijn studie, bestaande uit een
			demo applicatie voor een constraint-solving omgeving genaamd CHIP.}
			\medskip \\
			
		\end{tabular} \
% }}}

        \pagebreak

        \section*{Projecten}
        
        % Kennisnet
        \begin{tabular}{l p{10cm}}
            \tabelh{april 2005 -- juli 2005} & \tabelh{Zoekmachine voor het
            onderwijs} \\

            \omschrijving &
            Implementatie van een nieuwe versie van een zoekmachine voor het
            onderwijs. De zoekmachine bestaat uit een database van ruim 65000
            bronnen. Deze bronnen worden handmatig aangemeld, gecategoriseerd
            en gewaardeerd door een groot aantal redacteuren verspreid door
            het land.  In een team van 5 software engineers is de applicatie
            voorzien van een nieuwe webservice interface, een technisch beheer
            module en een vernieuwde redacteuren omgeving, compleet met
            autorisatie en workflow. \\
            \activiteiten & implementatie, refactoring, code review, testen.\\
            \tools & UML, MVC, Java 5, J2EE, Struts,
            Lucene, Servlets, JSP, HTML, CSS, \postgresql, Tapestry,
            Hibernate, Tomcat, Eclipse, Ant, JUnit, CVS \\
        \end{tabular}

        \medskip
       
        % TUMS
        \begin{tabular}{l p{10cm}}
            \tabelh{maart 2005 --  mei 2005} & \tabelh{Resource planning
            applicatie} \\
            \omschrijving & Ontwerp en realisatie van een resource
            planning applicatie. Het gaat om een legacy desktop applicatie die
            van een web interface moest worden voorzien. Na analyse van het
            bestaande systeem is besloten tot een volledige herontwerp van de
            applicatie, waarbij de nadruk wordt gelegd op onderhoudbaarheid en
            uitbreidbaarheid. Er is een 3-tier architectuur ontworpen,
            waarbij de functionaliteit zich concentreert in de business tier. \\
            \activiteiten & Analyse, technisch ontwerp, ontwikkeling. \\
            \tools & Java/J2EE, Hibernate, Tapestry,
            Servlets, HTML, CSS, Eclipse, Ant, \mysql, Subversion \\
        \end{tabular}

        \medskip
        
        % Nedal
        \begin{tabular}{l p{10cm}}
            \tabelh{december 2004 --  maart 2005} & \tabelh{Applicatie
            integratie m.b.v. een webservice} \\
            \omschrijving &
            Ontwerp en realisatie van een interface die de gegevensstroom
            realiseert tussen systemen in een produktieomgeving en het
            kantoor.  Omdat deze twee systemen door verschillende leveranciers
            op verschillende platformen zijn ontwikkeld, isgekozen voor een
            oplossing m.b.v. een webservice interface. \\
            \activiteiten: &
            Consultancy, Research, Technisch Ontwerp, Ontwikkeling, Integratie \\
            \tools & Java/J2EE, Webservices (JAX-RPC),
            Axis, Eclipse, Subversion \\
        \end{tabular}

        \medskip
        
        %Hydron
        \begin{tabular}{l p{10cm}}
            \tabelh{oktober 2004 -- januari 2005} & \tabelh{Webvisualisatie
            van actuele meetgegevens} \\
            \omschrijving &
            Ontwerp en realisatie van real-time visualisatie van actuele
            meetgegevens. De gebruiker kan lokaties op een geografische kaart
            aanklikken, waarbij aktuele meetgegevens getoond worden over
            produktie en transport op die lokatie. De actuele gegevens zijn
            afkomstig uit een reeds bestaand meetsysteem. De applicatie houdt
            tevens een historie bij van de ontvangen meetgegevens, welke
            d.m.v.  trendgrafieken bekeken kunnen worden. \\
            \activiteiten &
            Functioneel en technisch ontwerp, research, ontwikkeling,
            vormgeving, documentatie \\
            \tools &
            Java/J2EE,  JSP, Servlets, Hibernate, J-Integra, \mysql,
            JFree\-Chart, Tomcat, Eclipse, HTML, CSS. \\
        \end{tabular}

        \medskip

        %DZL
        \begin{tabular}{l p{10cm}}
            \tabelh{januari 2004 -- oktober 2004} & \tabelh{Prototype grafisch
            CMS} \\
            \omschrijving &
            Ontwerp en bouw van een prototype ter ondersteuning van een concept dat
            bedacht is door een klant vanuit de grafische industrie. Het gaat om een
            content management systeem waarin teksten en afbeeldingen kunnen worden
            opgeslagen. Afbeeldingen moeten op basis van bepaalde criteria kunnen worden
            gevonden en direct in verschillende formaten (grootte, resolutie,
            bestandsformaat etc.) kunnen worden opgeslagen afhankelijk van de toepassing
            (drukwerk, website etc.). Ook kan de gebruiker drukwerk genereren op basis van
            sjablonen (PDF), met variabele inhoud, dat direct naar de drukker
            kan. \\
            \activiteiten &
            Consultancy, ontwerp en ontwikkeling, vormgeving. \\
            \tools &
            Java,  MMBase, JSP, HTML, Eclipse, PDF \\
        \end{tabular}
       
        \medskip
        
        %Mercator
        \begin{tabular}{l p{10cm}}
            \tabelh{oktober 2003 -- april 2005} & \tabelh{CMS voor corporate
            websites} \\
            \omschrijving &
            Realisatie van nieuwe functionaliteiten in een open source content-management
            systeem. Betreft voornamelijk de ondersteuning van meerdere sites binnen één
            systeem, met daaraan gekoppeld autorisatie, workflow, gebruikersbeheer en
            navigatie. Bedoeling is een systeem te ontwikkelen ter ondersteuning van
            meerdere internet- en intranet sites, ieder met eigen look-and-feel, content
            en navigatiestructuur, waarbij beheer via een centraal CMS gebeurt. Het CMS
            kent een workflow met autorisatie voor verschillende gebruikers op
            verschillende content. \\
            \activiteiten &
            Analyse, functioneel en technisch ontwerp, research, ontwikkeling,
            documentatie \\
            \tools &
            Java/J2EE, Web-in-a-Box, MMBase, \mysql, Junit, Tomcat, Eclipse,
            Ant, JSP, XML, HTML, CSS. \\
        \end{tabular}

        \medskip
        
        % Emerson
        \begin{tabular}{l p{10cm}}
            \tabelh{augustus 2003 -- oktober 2003} & \tabelh{Aanpassingen legacy
            software} \\
            \omschrijving &
            Het inventariseren, analyseren en implementeren van de benodigde veranderingen
            in legacy software, geschreven voor een systeem dat van een oude naar een
            nieuwe omgeving wordt overgezet. Het voorbereiden en uitvoeren van
            systeemtests on-site. \\
            \activiteiten &
            Analyse, research, ontwikkeling \\
            \tools &
            FORTRAN-77 \\
        \end{tabular}
        
        \medskip
        
        %Vriendje Niels
        \begin{tabular}{l p{10cm}}
            \tabelh{januari 2003 -- februari 2003} & \tabelh{Proof-of-concept
            intranet site} \\
            \omschrijving &
            Het maken van een proof-of-concept demo implementatie van een intranet voor
            een dienstverlenende instantie in de sociale sector. Basis functionaliteiten
            zijn autorisatie, navigatie, smoelenboek, nieuws (incl. archief), kalender,
            WYSIWYG editors. \\
            \activiteiten & Consultancy, analyse, ontwerp, ontwikkeling,
            vormgeving. \\
            \tools & 
            MMBase, Tomcat, \mysql, Java, Eclipse, Ant \\
        \end{tabular}

        \medskip
        
        %FlexiForce
        \begin{tabular}{l p{10cm}}
            \tabelh{maart 2001 -- september 2002} &
            \tabelh{Produktconfiguratie-- en orderapplicatie} \\
            \omschrijving &
            Het ontwikkelen van een ordersysteem voor een
            leverancier/fabrikant van onderdelen voor overhead deuren. M.b.v.
            het systeem kan een complete deur geconfigureerd en besteld
            worden. Belangrijkste speerpunt is het feit dat het systeem zowel
            door de leverancier zelf als door de klant gebruikt kan worden.
            Het systeem is meertalig, ondersteunt meerdere databases, en heeft
            een interface waarbij orders automatisch kunnen worden besteld en
            verwerkt via internet e-mail.  \\ 
            \activiteiten & 
            Ontwerp en implementatie van modules, ontwikkeling van
            kernfunctionaliteit. \\
            \tools &
            Visual Basic, MS-Access, SQLServer, Ant \\
        \end{tabular}
      
        \medskip
        
        %Egel
        \begin{tabular}{l p{10cm}}
            \tabelh{mei 1994 -- september 1995} & \tabelh{Tentamensysteem} \\
            \omschrijving &
            Het ontwikkelen van een applicatie voor het administreren,
            samenstellen, genereren en afnemen van examens.  Functionaliteit
            omvat o.a. het beheren van een vragenbank met tentamenvragen,
            het samenstellen en automatisch genereren van tentamens op basis
            van uitgebreide criteria en het afnemen van de tentamens op de
            computer. Tentamens maken gebruik van multimedia zoals filmpjes
            en geluiden. Veiligheid en betrouwbaarheid was ook een sterk punt
            van aandacht bij het afnemen van de tentamens. \\
            \activiteiten &
            Functioneel en technisch ontwerp, software ontwikkeling. \\
            \tools &
            Visual Basic, MS-Access \\
        \end{tabular}
           
        \medskip
        
	\section*{Hobby's}
    \opt{en}{electronic music, retro computing}\opt{nl}{Elektronische muziek, retro computing.} \\
	\opt{en}{reading, movies.}\opt{nl}{Boeken, film.} \\
	\opt{en}{cooking \&\ eating.}\opt{nl}{Koken \&\ Eten.} \\
	\opt{en}{camping.}\opt{nl}{Kamperen.}

\end{document}
